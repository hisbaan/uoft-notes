\section*{Unit 2}

\setcounter{section}{2}
\setcounter{subsection}{0}

\subsection{The idea of limit --- (Non-rigorous) examples}

\underline{Example 1}:

\(\DS f(x) = \frac{x^{2} - 1}{x - 1}\)

\(f(1)\) is undefined.

We can simplify \(f\) as follows.

\[\frac{x^{2} - 1}{x - 1} = \frac{\br{x - 1} \br{x + 1}}{x - 1} = x + 1 \quad \fbox{\text{if} x \neq 1}\]

This means that \(f\) is just a line, but the point at \(x = 1\) is missing.

If we look at all of the values near \(f(1)\), we can see that they go toward \(2\). This is the informal idea of limit. The notation for this is \(\DS \lim_{x \to 1} f(x) = 2\) which is read as ``The limit as \(x\) approaches \(1\) of \(f(x)\) is \(2\).''

If \(x\) is very close to \(1\) (but \(x \neq 1\)), then \(f(x)\) is very close to \(2\).

\underline{Example 2}:

\(\DS f(x) = \frac{1 - \sqrt{1 + x}}{x}\).

Domain \(f = [-1, 0) \cup (0, \infty)\).

\(f(0)\) is undefined.

\begin{tabular}{c|c}
  \(x\) & \(f(x)\) \\
  \midrule
  1 & -0.4142135625 \\
  0.1 & -0.4880884817 \\
  0.01 & -0.4987562112 \\
  0.001 & -0.4996750625 \\
\end{tabular}

As \(x\) gets very close to \(0\), \(f(x)\) gets very close to \(-0.5\). If \(x\) is very close to \(0\) (but \(x \neq 0\)), then \(f(x)\) is very close to \(-0.5\).

Another way to understand this is some simplification. We can take \(f(x)\) and multiply and divide by the conjugate of the numerator.

\[\frac{1 - \sqrt{1 + x}}{x} = \frac{\br{1 - \sqrt{1 + x}} \br{1 + \sqrt{1 + x}}}{x \br{1 + \sqrt{1 + x}}} = \cdots = \frac{-1}{1 + \sqrt{1 + x}} \quad \fbox{\text{when} x \neq 0}\]

The new function has a vaule of \(-0.5\) at \(x = 0\). So even when \(f(0)\) is not defined, we can draw a conclusion as to what it is likely to be.

\underline{Summary}:

\(\DS \lim_{x \to a} f(x) = L\) roughly means that ``If \(x\) is close to \(a\) (but \(x \neq a\)), then \(f(x)\) is close to \(L\).''

\subsection{Examples of limits that do not exist}

\underline{Example 1}:

\(h(x) = \sin{\frac{\pi}{2x}}\)

\(h(0)\) is undefined

\begin{tabular}{c|c}
  \(x\) & \(f(x)\) \\
  \midrule
  1 & 1     \\
  1/2 & 0   \\
  1/3 & -1  \\
  1/4 & 0   \\
  1/5 & 1   \\
  1/6 & 0   \\
  1/7 & -1  \\
  1/8 & 0   \\
\end{tabular}

% TODO add graph

The function oscilates infinitely many times before reaching 0, so it never reaches 0. Similar to Achilles and the turtle (Zeno's paradox).

If \(x\) is close to \(0\), then \(h(x)\) is not close to one number.

We say that \(\DS \lim_{x \to 0} h(x) DNE\).

\underline{Example 4}:

\(F(x) = \frac{1}{\br{x - 1}^{2}}\).

\(F(1)\) is undefined.

This function has a vertical asymptote at \(x = 1\).

If \(x\) is close to \(1\) (but \(x \neq 1\)), then \(F(x)\) is very large.

We describe this case with the notation \(\DS \lim_{x \to 1} F(x) = \infty\). This also means, however, that the limit does not exist. The limit does not exist because for the limit to exist, \(F(x)\) must approach one single number but it is not approaching a single number as it is increasing to infinity.

% TODO add graph

\subsection{Side limits}

\underline{Example 1}:

\(\DS G(x) = \frac{x^{2} + x}{\abs{x}}\).

\(G(0)\) is undefined.

Since we are dealing with an absolute value here, we will break it into two cases.

If \(x > 0\), then \(\abs{x} = x\). This means that in this case, \(G(x) = \frac{x \br{x + 1}}{x} = x + 1\).

If \(x < 0\), then \(\abs{x} = -x\). This means that in this case, \(G(x) = \frac{x \br {x - 1}}{-x} = -x - 1\).

% TODO add graph

When \(x\) is close to \(0\), \(G(x)\) is not close to one single number, but instead two numbers. This means that \(\DS \lim_{x \to 0} G(x) \text{DNE}\).

We describe this case as a ``Side Limit.'' We say that \(\DS \lim_{x \to 0^{+}} G(x) = 1\) and that \(\DS \lim_{x \to 0^{-}} G(x) = -1\). The superscript denotes the side of the limit that we are approaching from.

\subsection{Distance and absolute values}

The best way to describe absolute value algebraicly is to break it into two cases.

For every \(x \in \R, \abs{x} = \begin{cases} x \quad & \text{if } x \geq 0 \\ -x \quad & \text{if } x < 0 \end{cases}\).

The best way to describe absolute value geometrically is as follows.

\begin{itemize}
  \item \(\abs{x}\) is the distance between \(x\) and \(0\).
  \item \(\abs{x - a}\) is the distance between \(x\) and \(a\).
\end{itemize}

Some properties of absolute values:

For every \(x, y \in \R\),

\begin{itemize}
  \item \(\abs{xy} = \abs{x} \abs{y}\)
  \item \(\abs{x + y} \leq \abs{x} + \abs{y}\)
\end{itemize}

\underline{Equivalent expressions: \(\abs{x - a} < \delta\)}:

\fbox{\color{red}The distance between \(x\) and \(a\) is smaller than \(\delta\)}

\begin{itemize}
  \item \(\abs{x - a} < \delta\)
  \item \(- \delta < x - a < \delta\)
  \item \(a - \delta < x < a + \delta\)
\end{itemize}

Any time you see one of these inequalities, you can move back and forth between these terms.

\subsection{The formal definition of limit}
\subsection{Limits at infinity}
\subsection{Prove a function has a limit from the definition --- Example 1}
\subsection{Prove a function has a limit from the definition --- Example 2}
\subsection{How to prove a limit DNE from the definition}
\subsection{Limit laws}
\subsection{Proof of the limit law for sums}
\subsection{The Squeeze Theorem}
\subsection{The definition of continuity}
\subsection{The main continuity theorem}
\subsection{Limits and composition of functions}
\subsection{Discontinuties (and how to remove them)}
\subsection{A geometric proof for a trigonometric limit}
\subsection{Review of basic techniques for computing limits}
\subsection{How to compute the limit of a rational function at infinity}
\subsection{The Extreme Value Theorem}
\subsection{The Intermediate Value Theorem}
