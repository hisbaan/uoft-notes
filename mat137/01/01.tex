\section*{Unit 1}

\setcounter{section}{1}

\subsection{Sets and notation}

A set is a ``collection of things'' (often numbers), called elements.
\begin{flalign} \nonumber
  &\begin{aligned}
    A &= \cbr{\text{even integers}} \\
    B &= \cbr{4, 5, 6} \\
    C &= \cbr{2, 4} \\
    D &= \underbrace{\cbr{4, 5}}_{\text{list of elements}}
  \end{aligned} &&
\end{flalign}

Set notation:

\begin{tabular}{c l l}
  Symbol        & Notation                  & Example                       \\
  \toprule
  \(\in\)       & ``is an element of''      & \(4 \in B\)                   \\
  \(\notin\)    & ``is not an element of''  & \(2 \notin B\)                \\
  \(\subseteq\) & ``is a subset of''        & \(D \subseteq B\)             \\
  \(\cup\)      & ``union of sets''         & \(C \cup D = \cbr{2, 4, 5}\)  \\
  \(\cap\)      & ``intersection of sets''  & \(C \cap D = \cbr{4}\)        \\
  \(\emptyset\) & ``empty set''             & \(\emptyset = \cbr{}\)        \\
\end{tabular} \\

Some important sets:

\begin{tabular}{l l}
  Naturals:     & \(\N = \cbr{0, 1, 2, 3, \dots}\)                          \\
  Integers:     & \(\Z = \cbr{\dots, -3, -2, -1, 0, 1, 2, 3, \dots}\)       \\
  Rationals:    & \(\Q = \cbr{\text{quotients of integers (fractions)}}\)   \\
  Reals:        & \(\R = \cbr{\text{numbers with a decimal expansion}}\)    \\
\end{tabular}

\subsection{Set-building notation}

\begin{enumerate}
  \item[1.]
        \(A = \overbracket{\cbr{ x \in \Z : x^{2} < 6 }}^{\text{description of the set}}\)

        \(A = \cbr{ x \in \Z \mid x^{2} < 6 }\)

        The part before the \(:\) or \(\mid\) is the group that we take elements from and the part after the \(:\) or \(\mid\) are extra constraints.

        This means that \(A = \cbr{-2, -1, 0, 1, 2}\). While we can describe \(A\) more easily here, there are times when we cannot be explicit, but we can still use set-building notation to describe the set.

  \item[2.]
        \(A = \cbr{-2, -1, 0, 1, 2}\)

        \(B = \cbr{2x \mid x \in A}\)

        In this example, again, the \(\mid\) means ``such that'' but on the left, we describe what elements in \(B\) look like and on the right,  we exmplain the notation that we used on the left.

        The sentence can be read as ``\(B\) is the set of elements of the form \(2x\) such that \(x\) is an element of \(A\).'' In other words, \(B\) consists of any element that is \(2\) times an element in \(A\). This means that \(B = \cbr{-4, -2, 0, 2, 4}\)
\end{enumerate}

\underline{Intervals}:

Let \(a, b \in \R\)
\begin{flalign} \nonumber
  &\begin{aligned}
    \quad   &1.~\sbr{a, b}      &&= \cbr{x \in \R \mid a \leq x \leq b} \\
            &2.~\br{a, b}       &&= \cbr{x \in \R \mid a < x < b} \\
            &3.~[a, b)          &&= \cbr{x \in \R \mid a \leq x < b} \\
            &4.~\br{a, \infty}  &&= \cbr{x \in \R \mid a < x} \\
            &5.~(-\infty, b]    &&= \cbr{x \in \R \mid x \leq b} \\
  \end{aligned} &&
\end{flalign}

\subsection{Quantifiers}

\(\forall =\) ``for all/every''

\(\exists =\) ``there exists/is'' {\color{cyan}(at least one)}

\begin{enumerate}
  \item[Ex: 1.] For all
        \begin{flalign} \nonumber
          &\begin{aligned}
            \forall x \in \R, &~x^{2} \geq 0 && \quad {\color{red} \text{True}} \\
            \forall x \in \R, &~x^{2} > \pi && \quad {\color{red} \text{False (\(x = 1\))}} \\
          \end{aligned} &&
        \end{flalign}
  \item[2.] There Exists
        \begin{flalign} \nonumber
          &\begin{aligned}
            \exists x \in \R \text{ such that } &~x^{2} = 5 && \quad {\color{red} \text{True (\(x = - \sqrt{5}\))}} \\
            \exists x \in \R \text{ such that } &~x^{2} = -1 && \quad {\color{red} \text{False}} \\
          \end{aligned} &&
        \end{flalign}
  \item[3.] Other
        \begin{flalign} \nonumber
          &\begin{aligned}
            x^{2} = 5 && \quad {\color{red} \text{meaningless}}
          \end{aligned} &&
        \end{flalign}
\end{enumerate}

\subsection{Double quantifiers}

\begin{enumerate}
  \item[1.] \(\forall x \in \Z, \exists y \in \Z \st x < y\)

        We are allowed to use a different \(y\) for each \(x\).

        For each \(x\) that we choose, there is a \(y\) such that the statement \(x < y\) is true.

        ``Every integer is smaller than some other one''

        This statement is true.

  \item[2.] \(\exists y \in \Z, \forall x \in \Z \st x < y\)

        We are only allowed to use a single \(y\).

        There exists a \(y\) for all possible \(x\) such that the statement \(x < y\) is true.

        ``There is an integer, \(y\), greater than all integers.''

        This statement is false.
\end{enumerate}

The order quantifiers are listed in matters a lot.

\subsection{Simple proofs with quantifiers}

\underline{Theorem 1}:

Let \(A = \cbr{2, 3, 4}\). \(\forall x \in A, x > 0\).

\begin{proof} \(\)

  \(2 > 0~\checkmark\)

  \(3 > 0~\checkmark\)

  \(4 > 0~\checkmark\) \(\qedhere\)
\end{proof}

\underline{Theorem 2}:

\(\forall x \in \Z, x > 0\)

\begin{proof} \(\)

  This theorem is false. To say that it is false is to say that its negation (opposite) is true. Therefore, to say that the theorem is false, we need to prove the following.

  \(\exists x \in \Z \st x \leq 0\)

  Take \(x = -1\). Since \(-1 \in \Z\) and \(-1 \leq 0\), we have proved that \emph{Theorem 2} is false. \(\qedhere\)
\end{proof}

\underline{Theorem 3}:

\(\forall x \in \Z, \exists y \in \Z \st x < y\)

\begin{proof} \(\)

  Let \(x \in \Z\)

  Take \(y = x + 1\)

  Since \(x \in \Z\), we know that \(x + 1 \in \Z\) and thus \(y \in \Z\). Since \(y = x + 1\) and \(x < x + 1\), we know that \(x < y\) as needed. \(\qedhere\)
\end{proof}

\subsection{Quantifiers and the empty set}

True or False?
\begin{flalign} \nonumber
  &\begin{aligned}
&\quad 1.~\forall x \in \emptyset, x > 0 && \text{\color{red} True}\\
&\quad 2.~\exists x \in \emptyset \st x > 0 && \text{\color{red} False}\\
  \end{aligned} &&
\end{flalign}
For the following, we will abbreviate \(x > 0\) as *
\begin{itemize}
  \item To prove (1) is true, I need to verify \underline{all} elements in \(\emptyset\) satisfy *. {\color{red}\(\checkmark\)}
  \item To prove (1) is false, I need to find \underline{one} element in \(\emptyset\) doesn't satisfy *. {\color{red}\(\times\)}
\end{itemize}

Since there are no elements in \(\emptyset\), we can show that all of the elements in \(\emptyset\) satsify almost any rule. Because of this, generally, any statment which starts with \(\forall x \in \emptyset\) will be true.

\subsection{Conditional statements}

A conditionial statment is something that has the word ``If''. The following three statements are all different ways of saying the same thing.
\begin{itemize}
  \item If \(P\), then \(Q\)
  \item \(P \implies Q\)
  \item \(P\) implies \(Q\)
\end{itemize}

When this statement is true, whenever \(P\) is true, \(Q\) must be true as well. Whenever \(P\) is false, we don't care. Thus to prove an implication that is not known to be true, assume \(P\) is true and prove \(Q\).

\underline{Example 1}:

Let \(x \in \R\).

\begin{tabular}{l c c c }
              & \(x > 10\)        & \(\implies\)  & \(x > 6\) \\
  \(x = 12\)  & {\color{green}T}  &               & {\color{green}T}  \\
  \(x = 8\)   & {\color{red}F}    &               & {\color{green}T}  \\
  \(x = 12\)  & {\color{red}F}    &               & {\color{red}F}    \\
\end{tabular}

Whenever \(P\) is true, \(Q\) is true as well. Whenever \(P\) is false, we can't say anything about \(Q\), it may be true or false.

\underline{Example 2}:

Let \(A \subseteq \R\). Assume we know \(x \in A \implies x > 0\). What can we conlcude in the following scenarios?

\begin{itemize}
  \item \(x \notin A\) \qquad {\color{red}No conclusion.}
  \item \(x > 0\) \qquad {\color{red}No conclusion.}
  \item \(x \leq 0\) \qquad {\color{red}We can conclude \(x \notin A\)}
\end{itemize}

\(P \implies Q\) and \(\neg Q \implies \neg P\) both mean the same.

\underline{Example 3}:

Let \(n \in \Z\)
\begin{itemize}
  \item \(n\) is even \(\impliedby n\) is a multiple of 4
  \item \(n\) is even \(\iff\) \(n + 1\) is odd
\end{itemize}

The \(\iff\) symbol is called ``if and only if'', commonly abbreviated to ``iff''. It is an implication in both directions. \(P \iff Q \) means \(P\) is true if and only if \(Q\) is true. They must both be true or both be false.

\underline{Example 4}:

True or False? \(0 > 1 \implies 103574289 \text{ is prime}\).

We know that the hypothesis of the implication is false, but we do not know whether the conclusion is true or false. But for the whole implication, we do not need to know whether the number is prime or not. The entire statement must be true as an implication is only false when the hypothesis is true and the conclusion is false. This is called a vacous truth.

\subsection{How to negate a conditional statement}

Let \(A \subseteq \R\). If \(x \in A\), then \(x > 0\).

This can be rewritten as \(x \in A \implies x > 0\).

Let us break down what this statement means.

\(\forall x \in A \begin{cases} x \in A \tand x > 0 \\ x \notin A \tand x > 0 \\ x \notin A \tand x \leq 0 \end{cases}\)

The negation of this statement would the permutation that is not one of the three above cases. This means that it would be \(\exists x \in \R \st x \in A \tand x \leq 0\).

\subsection{A bad proof}

``\underline{Theorem}'': \(\DS \sqrt{xy} \leq \frac{x + y}{2}\)

``\underline{Pf}'':

\begin{flalign} \nonumber
  &\begin{aligned}
    xy &\leq \br{\frac{x + y}{2}}^{2} && {\color{red}\text{I start by assuming what I want to prove. This is very wrong}} \\
    xy &\leq \frac{x^{2} + 2xy + y^{2}}{4}\\
    4xy &\leq x^{2} + 2xy + y^{2} \\
    0 &\leq x^{2} - 2xy + y^{2} \\
    0 &\leq \br{x - y}^{2} \\
  \end{aligned} &&
\end{flalign}

Because of the incorrect structure of the proof, I did not realize here that both \(x\) and \(y\) must be positive for the inital statement to be true. The goal of a proof is to determine and show whether a statement is true or not, not to make a statement true by any means necessary. Finally, this proof contains no words, only algebra. For some proofs this is okay, but most proofs should contain an explination of why you are doing things.

Let's fix this proof.

\underline{Theorem}:

Let \(x, y \geq 0\). Then \(\DS \sqrt{xy} \leq \frac{x + y}{2}\).

\begin{proof} \(\)

  Since a square is always non-negative, we know that \(0 \leq \br{x - y}^{2}\). We can manipulate this as follows.

  \begin{flalign} \nonumber
    &\begin{aligned}
      0 &\leq \br{x - y}^{2} \\
      0 &\leq x^{2} - 2xy + y^{2} \\
      0 &\leq x^{2} + 2xy - 4xy + y^{2} \\
      4xy &\leq x^{2} + 2xy + y^{2} \\
      xy &\leq \frac{x^{2} + 2xy + y^{2}}{4} \\
      \sqrt{xy} &\leq \sqrt{\frac{x^{2} + 2xy + y^{2}}{4}} && \text{(We can only take the squre root because both sides are non-negative)} \\
      \sqrt{xy} &\leq \frac{\sqrt{\br{x + y}^{2}}}{2} \\
      \sqrt{xy} &\leq \abs{\frac{x + y}{2}} \\
      \sqrt{xy} &\leq \frac{x + y}{2} \qedhere && \text{(Since \(x, y \geq 0\))}
    \end{aligned} &&
  \end{flalign}
\end{proof}

It is important to note that the previous bad ``proof'' was not entirely useless. It is good rough work to help us come up with ideas on how to prove a theorem. For example, the last line in the bad ``proof'' is the first line in the good proof.

\subsection{How to write a rigorous, mathematical definition}

\subsection{Proofs: an example}

\subsection{Proofs: a non-example}

\subsection{Proofs: a theorem}

\subsection{Proof by induction}

\subsection{One Theorem. Two Proofs.}
